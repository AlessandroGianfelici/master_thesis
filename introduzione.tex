\chapter*{Introduction}
\noindent
The quantum field theory is one of the most successful frameworks for physical mathematical modeling
developed by the XX century physicists. 
Both quantum and statistical physics are described in terms of fields in their present formulations and can be mostly studied with the same mathematical tools.
The Standard Model of fundamental interactions, which encodes all our present knowledge about elementary particles and fundamental forces, is itself an interacting quantum field theory.
It is a common belief in the theoretical physicist community that every kind of fundamental physical
phenomenon will be reduced,  in the future, to a field theory of some kind, even if maybe not as local quantum field theories.
Indeed also string theory is still waiting to rise to a field theoretical description, being the attempts to formulate a string field theory not yet very successful.

Renormalization is one of the central tools on which almost every field theory is built, because it allows to coherently derive misurable quantity from the theory and relate phenomena which appears at different scales of observation.
Indeed this is the key observation at the base of its modern formulation and comprehension, and in particular it helps in relating the microscopic behavior of a system with its long distance behavior.
The modern paradigm of renormalization was developed by several people among whom K. Wilson gave the strongest impulse in the seventies, leading to the so called Wilson's renormalization (semi)group idea and to the development of some tools which, in principle, can permit a nonperturbative analysis of quantum and statistical systems.

Following Wilson's philosophy of integrating the quantum (or thermal, in it's statistical physics application) fluctuations, for example momentum shell by momentum shell, several formulations of the nonperturbative renormalization group have been developed since then, among which the Wetterich's formulation, based on the concept of a scale dependence of the effective (average) action, is one of the most employed today, due to the simplicity of its application with respect to other
approaches.

One of the interesting features of this approach is that, in principle and in practice, one can also find a systematic way to obtain the perturbative results which are most commonly extracted using the standard perturbative approach. But at the same time it can go beyond it. What is still missing is a way to gibe precise estimates of the errors associated to a give ``truncation'' and renormalisation scheme.

At non perturbative level this approach is been currently used to study some of the difficult problems in theoretical physics. One is the confinement phase of Quantum Chromodynamics (QCD), and currently for several observables one can obtain results at least at the same order of accuracy of the one derived in a lattice formulation.
The other is the study of gravitational interactions.
It is well known that General Relativity can not admit within perturbation theory a coherent quantum formulation in 
terms of a quantized field, because of the divergences that arise in quantum computations which cannot be absorbed by a redefinition of the fields or of the coupling constants. 
A simple power counting criterion already shows that the General Relativity as a QFT is perturbatively non renormalizable at two loops, since the Newton constant has mass dimension $-2$, while gravity interacting with matter is not renormalizable already at one loop.

In view of that, it is clear that nonperturbative methods are the ideal candidates in order to provide predictions of the UV behavior of this theory. Indeed in 1976 Weinberg proposed a generalization of the concept of renormalizability, based on the non trivial fixed point structure of the underlying renormalization group flow. That was called asymptotic safety. 

The idea, quoting the words of Weinberg himself, is that: ``A theory is said to be asymptotically safe if the essential coupling parameters approach a fixed point as the momentum scale of their renormalization point goes
to infinity''. The parameters, made dimensionless by rescaling, should stay finite in this limit and reversing the flow from the ultraviolet (UV) to the infrared (IR) regime, only a finite number of them should flow away from the UV limiting value (relevant directions).

This thesis is devoted to the application of Wetterich's nonperturbative functional renormalization method to the physics of scalar quantum field theory with an internal $O(N)$ symmetry, both in a flat Euclidean spacetime and in the case of the coupling to a gravitational field.
In flat space, \emph{i.e.} with no gravitational interactions, we shall address the formulation in the first two order of the so called derivative expansion, and mostly derive and discuss some aspects of the flow equations
in the limit $N \to \infty$, where some results have already ben derived in the literature and other conjectured.

In the case of the coupling to a dynamical gravitational field, whose quantum nature has been considered as a QFT in the paradigm of the asymptotic safety, we attempt to derive with some approximation scheme the flow equations for a three dimensional euclidean space time.
Then the fixed point equations have been analysed in the quest of searching the scaling solutions at criticality as a function of N.


This thesis is organized as follows:
\begin{enumerate}
 \item In Chapter 1 I have exposed some basic concepts of the path integral formulation of a quantum field theory and some useful notations have been introduced.
 \item In Chapter 2 I have shown how the Wetterich approach to functional renormalization theory allows a generalization of the concepts exposed in the first chapter
 in terms of running (\emph{i.e.} scale dependent) objects. I have also shown in detail how this renormalization technique can be applied to a simple QFT model, the scalar field in $D$ dimensions.
 \item In Chapter 3 I have considered the scalar linear $O(N)$ model in $D$ dimensions, truncated at the second order in the derivative expansion. I have derived analytically 
 the flow equations for the relevant quantities, considering then the special case $N\to\infty$.
 \item In Chapter 4 the analysis of the fixed point structure of the theory has been begun. First the case of constant wavefunction renormalization and of 
 a vanishing anomalous dimension, already known in the literature \cite{vaccascaling}, has been exposed. Then we have considered the NLO case, both for a vanishing and
 a non vanishing anomalous dimension. 
 \item In Chapter 5 I have introduced the minimal coupling to the gravitational fields 
(treated at the quantum level with a QFT in the paradigm of the 
\emph{asymptotic safety}) and I have studied the fixed point structure of the model.
\end{enumerate}
