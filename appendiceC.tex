\chapter{Threshold functions}
In this appendix I will define the objects known in the literature as \emph{threshold functions}, which will allow us to express in a more compact and elegant way the flow equations for the relevant observable of the $O(N)$ model, $U_k(\rho)$, $Z_k(\rho)$ and $Y_k(\rho)$.

For our model three different types of threshold functions are defined:

\begin{enumerate}
 \item $$\L^D_{mn} = -\int_0^\infty dy\widetilde{\partial}_y\left[y^{\frac{D}{2}-1}(g_\perp(y))^m(g_\parallel(y))^n\right]$$
 \item $${\M^D_{mn}} = -\int_0^\infty dy\widetilde{\partial}_y\left[y^{\frac{D}{2}}(g'_\perp(y))^2) (g_\perp(y))^{m-4}(g_\parallel(y))^n\right]$$
 \item $$\widetilde \M^D_{mn} = -\int_0^\infty dy \widetilde{\partial}_y\left[y^{\frac{D}{2}}(g'_\parallel(y))^2) (g_\perp(y))^{m}(g_\parallel(y))^{n-4}\right]$$
 \item $$\N^D_{mn} = \int_0^\infty dy\widetilde{\partial}_y\left[y^{\frac{D}{2}}g'_\perp(y)(g_\perp(y))^{m-2}(g_\parallel(y))^n\right]$$
 \item $$\widetilde \N^D_{mn} = \int_0^\infty dy\widetilde{\partial}_y\left[y^{\frac{D}{2}}g'_\parallel(y)(g_\parallel(y))^{m-2}(g_\perp(y))^n\right]$$
 \item $$ Q^{d,\alpha}_{n,m} = \frac{n-2}{2D} M^{D + 2\alpha}_{n+1,m} + \frac{2m}{D} \Big(M^{D + 2\alpha}_{n,m+1}+ N^{D + 2\alpha}_{n,m+1}\Big) - \frac{2\alpha}{D}N^{D + 2\alpha-2}_{n,m}$$
 \item $$ \widetilde Q^{d,\alpha}_{n,m} = \frac{n-2}{2D}\widetilde  M^{D + 2\alpha}_{n+1,m} + \frac{2m}{D} \Big(\widetilde M^{D + 2\alpha}_{n,m+1}+\widetilde  N^{D + 2\alpha}_{n,m+1}\Big) - \frac{2\alpha}{D}\widetilde N^{D + 2\alpha-2}_{n,m}$$
 \end{enumerate}
Where I used the definitions of the longitudinal and trasversal projections of the dimensionless exact propagators, that I recall here:
$$g_\perp(y) = \frac{1}{u_k'(\widetilde{\rho}) + [z_k(\widetilde{\rho}) + r_k(y)]y}$$
$$g_\parallel(y) = \frac{1}{u_k'(\widetilde{\rho}) + 2\widetilde{\rho}u_k''(\widetilde{\rho})+ [z_k(\widetilde{\rho}) + \widetilde{\rho}\mathcal{Y}_k(\widetilde{\rho})+ r_k(y)]y}$$
The derivative $\widetilde{\partial}_y$ is an object, sometimes used in literature (see, for example, \cite{dtilde} or \cite{delamotteintr}) that acts as a ``derivative''
on the renormalized cutoff function $r_k(y)$, giving the result:
$$\widetilde{\partial}_y r(y) = -(2y\partial_y + \eta_k)r(y)$$
and leave invariant the other $y$-dependent observables in the integrand of the threshold functions.

The definitions of the threshold functions, despite its advantages, presents some difficulty when the integrands of these functions has to be explicitly calculated using a numerical routine, due to the definition of $\widetilde{\partial}_y$.
In order to circumvent it, a wide used trick is to give two different names to $y$, so it will be either called $y$ or $\widetilde{y}$. We then define a function $r(y, \widetilde{y})$ such that:
\begin{enumerate}
 \item $$\left. r(y, \widetilde{y})\right|_{\widetilde{y} = y} = r(y)$$ 
 \item $$\partial_{\widetilde{y}} r(y, \widetilde{y}) = \widetilde{\partial}_y r(y) = -(2y\partial_y + \eta_k)r(y)$$
 \item $$\partial_y r(y, \widetilde{y}) = \partial_y r(y)$$
\end{enumerate}














