
   %%%%%%%%%%%%%%%%%%%%
   %                  %
   %  appendiceD.tex  %
   %                  %
   %%%%%%%%%%%%%%%%%%%%
   
   
\chapter{Conventions and formulas for the gravitational coupling}   
In this appendix I will expose some definition and I will derive some formulas useful in order to fully 
understand the calculations done in the fifth chapter of this thesis.

\section{York decomposition}
The York decomposition is an useful way to decompose symmetric traceless two index tensor $\widetilde{h}_{\mu\nu}$:
\be
\label{york}
\widetilde{h}_{\mu\nu}=\tth_{\mu\nu}
+\bnabla_\mu\xi_\nu
+\bnabla_\nu\xi_\mu
+\bnabla_\mu\bnabla_\nu\sigma
-\frac{1}{d}\bg_{\mu\nu}\bnabla^2\sigma+\frac{h}{d}\bg_{\mu\nu} \\ ,
\ee
where $\tth_{\mu\nu}$ is a two index tensor which satisfies:
\begin{equation}
 \bnabla^\mu \tth_{\mu\nu} = 0
\end{equation}

\section{Calculation of $\sqrt{g}$}
In order to slit the full spacetime metric $g_{\mu\rho}$ in a fixed background $\bg_{\mu\rho}$ and a quantum fluctuation, I have used the 
following exponential parametrization:
\begin{equation}
\label{decomp}
g_{\mu\nu}=\bg_{\mu\rho}(e^h)^\rho{}_\nu
\end{equation}
following what has been done in some recent papers (see, for example, \cite{VACCA52}, \cite{VACCA28} and \cite{VACCA27}).
The background metric $\bg$ is the one that will be used to raise and lower indices, so we can define:
$$h_{\mu\nu}=\bg_{\mu\rho}h^\rho{}_\nu$$
that reveals to be a symmetric tensor.

The full metric is thus expressed ad a power series of the quantum fluctuation tensor $h_{\mu\nu}$, 
in the following way:
\bea
g_{\mu\nu}&=&\bg_{\mu\nu}+h_{\mu\nu}
+\frac{1}{2}h_{\mu\lambda}h^\lambda{}_\nu+\ldots
\\
g^{\mu\nu}&=&\bg^{\mu\nu}-h^{\mu\nu}
+\frac{1}{2}h^{\mu\lambda}h_\lambda{}^\nu+\ldots
\eea
We note that here both the covariant and the contravariant metrics are nonpolynomial in the quantum 
fluctuation, in contrast to what happens using the usual linear split:
$$g_{\mu\nu}=\bg_{\mu\nu}+h_{\mu\nu}$$
The linear terms are the same as in the linear parametrization,
some differences appears at the second order of the expansion. 

Another significant difference is that,
due to the formula: 
\begin{equation}
 \det e^h=e^{\tr h}
\end{equation}
only the traceless part of $h$
enters in the definition of the determinant of the full metric, at all orders.

For which it's convenient to split the fluctuation tensor $h_{\mu\nu}$ into a traceless part $\tilh$ and 
a pure trace part:
\be
h^\mu{}_\nu=\tilh^\mu{}_\nu+\frac{h}{d}\delta^\mu_\nu
\ee
where I have used the notation $\Tr h \equiv h$.
%(we use ``$\tau$'' instead of the natural symbol ``$h$'' because the latter is often used for the
%whole matrix  $h^\mu{}_\nu$).
Then the determinant of the full metric can be expressed as a power series of the trace of the fluctuation:
\be
\sqrt{g}=e^{\frac{h}{2}}\sqrt{\bg}=\sqrt{\bg}\left(1+\frac{h}{2}+\frac{h^2}{8}+\ldots\right)\ .
\ee
where I have indicated with $\bar{g}$ the determinant of the background metric.



\section{Hessian of a scalar $O(N)$ field coupled to gravity}
In this section I will show explicitly the procedure that leads to the  
expandsion of the effective average action up to the second order in the fluctuations I used in order to 
find the Hessian of the model.

I recall the form we have hypothesized for the effective action \eqref{ONaction}:
\begin{equation}
\label{ONaction}
\Gamma_k [\phi, g] = \int {d}^dx\sqrt{g}\left(U(\rho)+\frac{1}{2}\bg^{\mu\nu} \partial_\mu\phi^a \partial_\nu \phi_a-F(\rho)R\right)
\end{equation}
First of all I will obtain the expansion of the Ricci scalar $R$.
For the Christoffel symbols we have:
\bea\label{g2}
\hat\G^\a_{\mu\nu}
&=& \bar \G^\a_{\mu\nu} + \hat\G^{\a (1)}_{\mu\nu} + \hat\G^{\a(2)}_{\mu\nu},
\eea
where
\bea\label{g3}
\hat\G^{\a (1)}_{\mu\nu} &=& \frac12\left(\bnabla_\nu h^\a{}_\mu
 +\bnabla_\mu h^\a{}_\nu-\bnabla^\a h_{\mu\nu}\right), \\
\hat\G^{\a(2)}_{\mu\nu} &=& -\frac12 h^{\a\b} (\bnabla_\nu h_{\mu\b}
+\bnabla_\mu h_{\nu\b}-\bnabla_\b h_{\mu\nu})\\\label{g1}
&&+\frac{1}{4}\left(\bnabla_\mu(h^{\alpha\lambda}h_{\lambda\nu})
+\bnabla_\nu(h^{\alpha\lambda}h_{\lambda\mu})-\bnabla^\alpha(h_\mu{}^\lambda h_{\lambda\nu})\right)\nonumber
\eea
The Ricci curvature tensor is given by:
\begin{equation}\label{ricci}
{R^\mu}_{\nu\mu\sigma} =
\partial_{\mu}{\Gamma^\mu_{\sigma\nu}} - \partial_{\sigma}\Gamma^\mu_{\mu\nu} 
+ \Gamma^\mu_{\mu\lambda} \Gamma^\lambda_{\sigma\nu}
- \Gamma^\mu_{\sigma\lambda}\Gamma^\lambda_{\mu\nu}
\end{equation}
so, substituting the equations \eqref{g1},\eqref{g3} and \eqref{g2} into \eqref{ricci}, the expression of the Ricci tensor at the second order in $h_{\mu\nu}$ reads:
\be
\hR_{\nu\sigma}=\hR_{\mu\nu}{}^\mu{}_\sigma = 
\bR_{\mu\nu}{}^\mu{}_\sigma 
+ \hR^{(1)}_{\mu\nu}{}^\mu{}_\sigma
+ \hR^{(2)}_{\mu\nu}{}^\mu{}_\sigma, 
\ee
where
\begin{equation}
 \hR^{(1)}_{\mu\nu}{}^\mu{}_\sigma = \bnabla_\mu\hat\G^{(1)\mu}_{\nu\sigma}-\bnabla_\nu\hat\G^{(1)\mu}_{\mu\sigma}
\end{equation}

\begin{equation}
 \hR^{(2)}_{\mu\nu}{}^\mu{}_\sigma = \bnabla_\mu\hat\G^{(2)\mu}_{\nu\sigma}-\bnabla_\nu\hat\G^{(2)\mu}_{\mu\sigma}+[\hat\G^{(1)}_\mu,\hat\G^{(1)}_\nu]^\mu{}_\sigma\ .
\end{equation}

Finally one can combine the expansion of $\sqrt{g}$ with that of $R$
and integrate over spacetime.

We can then rewrite, up to an inifluent total derivative term, the second order terms in the expansion of the Hilbert action in the following way:
\bea
\int d^Dx\sqrt{\bg}\Bigl[&&\!\!\!\!\!\!
\frac{1}{4}h_{\mu\nu}\bnabla^2 h^{\mu\nu}
-\frac{1}{2}h_{\mu\nu}\bnabla^\mu\bnabla^\rho h_\rho{}^\nu
+\frac{1}{2} h\bnabla_\mu\bnabla_\nu h^{\mu\nu}
-\frac{1}{4} h\bnabla^2 h+
\nonumber\\
&&
+\frac{1}{2}\bR_{\mu\rho\nu\sigma}h^{\mu\nu}h^{\rho\sigma}
-\frac{1}{2}\bR_{\mu\nu}h^{\mu\nu}h
+\frac{1}{8}\bR h^2\Bigr]
\eea
Now, using the York decomposition on the metric and using the following relations, which hold on the sphere $S^d$:
\bea
\int dx \sqrt{\bg}\, h_{\mu\nu}\bnabla^2 h^{\mu\nu} &=& \int dx \sqrt{\bg} \Big[
\tth_{\mu\nu} \bnabla^2 \tth^{\mu\nu} 
-2\xi_\mu\left(\bnabla^2+\frac{\bR}{d}\right)\left(\bnabla^2+\frac{D+1}{D(D-1)}\bR\right)\xi^\mu+
\nonumber\\
&& 
+\frac{D-1}{D}\sigma\bnabla^2\left(\bnabla^2+\frac{2\bR}{D-1}\right)
\left(\bnabla^2+\frac{\bR}{D-1}\right)\sigma 
+ \frac{1}{D} h\bnabla^2h\Big]\ ,
\nonumber\\
%
\int dx \sqrt{\bg}\,h_{\mu\nu} \nabla^\mu\nabla_\rho h^{\rho\nu} &=& 
\int dx \sqrt{\bg}\Big[-\xi_\mu\left(\bnabla^2+\frac{\bR}{D}\right)^2\xi^\mu 
+\frac{(D-1)^2}{D^2}\sigma \bnabla^2\left(\bnabla^2+\frac{\bR}{D-1}\right)^2\sigma+
\nonumber\\
&& +\frac{2(D-1)}{D^2} h \bnabla^2\left(\bnabla^2+\frac{\bR}{D-1}\right)\sigma 
+\frac{1}{D^2} ( h)\bnabla^2 h\Big]\ ,
\nonumber\\
%\int dx \sqrt{\bg} h \nabla^\mu\nabla^\mu h_{\mu\nu} &=& \int dx \sqrt{\bg} \left[ \frac23 h \bnabla^2 \left(\bnabla^2 + \frac{R}{2}\right) \sigma +\frac13 h\bnabla^2 h \right]\ ,
%\nonumber\\
\int dx \sqrt{\bg}\,h_{\mu\nu} h^{\mu\nu} &=& 
\int dx \sqrt{\bg}
\Big[\tth_{\mu\nu}\tth^{\mu\nu} 
+2\xi_\mu \left(-\bnabla^2-\frac{\bR}{D}\right) \xi^\mu +
\nonumber\\
&& +\frac{D-1}{D}\sigma\bnabla^2\left(\bnabla^2+\frac{\bR}{D-1}\right)\sigma  
+\frac{1}{D}h^2\Big]\ .
\label{hterms}
\eea

Collecting all terms we can rewrite the quadratic effective action in terms
of the independent fields $\tth$, $\xi$, $\sigma$, $h$ and $\delta\phi$:
\bea
&&\int dx\sqrt{\bg}\Biggl[
F(\bphi)\Biggl(
\frac{1}{4}\tth_{\mu\nu}\left(-\bnabla^2+\frac{2\bR}{D(D-1)}\right)\tth^{\mu\nu}
-\frac{(D-1)(D-2)}{4D^2}\sigma'\left(-\bnabla^2\right)\sigma'
\nonumber\\
&&
-\frac{(D-1)(D-2)}{2D^2}h \sqrt{(-\bnabla^2)\left(-\bnabla^2-\frac{\bR}{D-1}\right)}\sigma'
-(D-1)(D-2) \frac{h}{2D} \left(-\bnabla^2+\frac{(D-2)\bR}{2(D-1)}\right)\frac{h}{2D}\Biggr)
\nonumber\\
&&
-F'(\bphi)\frac{D-1}{D} \delta\phi \left(
\sqrt{(-\bnabla^2)\left(-\bnabla^2-\frac{\bR}{D-1}\right)}\sigma'
+2D\left(-\bnabla^2+\frac{(D-2)\bR}{2(D-1)}\right)\frac{h}{2D}
\right)
\nonumber\\
&&
+\frac{1}{2} \delta\phi (-\bnabla^2+V''(\bphi)-F''(\bphi)\bR)\delta\phi
+\frac{1}{2}V'(\bphi)h \delta\phi
+\frac{1}{8}V(\bphi)h\Biggr]
\eea
We note that the kinetic operator of the $h$ field is the
conformal scalar operator.


\section{Transformation properties}
In this section I will discute the behavior of the metric under gauge transformation.
Under an infinitesimal diffeomorphism $\epsilon$, the metric transformation is given by the Lie derivative:
\be
\label{transfg}
\delta_\epsilon g_{\mu\nu}=
\Lie_\epsilon g_{\mu\nu}
\equiv
\epsilon^\rho\partial_\rho g_{\mu\nu}
+g_{\mu\rho}\partial_\nu\epsilon^\rho
+g_{\nu\rho}\partial_\mu\epsilon^\rho\ .
\ee
Now we have to define the transformations of $\bg$ and $h$ in such a way that the full metric
defined in \eqref{decomp} transforms according to \eqref{transfg}.
The simplest one is the background transformation.
If we treat $\bg$ and $h$ as tensors under $\delta_\epsilon$:
\be
\delta^{(B)}_\epsilon\bg_{\mu\nu}=\Lie_\epsilon \bg_{\mu\nu}\ ;
\qquad
\delta^{(B)}_\epsilon h^\mu{}_\nu=\Lie_\epsilon h^\mu{}_\nu\ .
\ee
then we have also:
\be
\delta^{(B)}_\epsilon(e^h)^\mu{}_\nu=\Lie_\epsilon(e^h)^\mu{}_\nu
\ee
and \eqref{transfg} follows.

The ``quantum'' gauge transformation of $h$ is defined so as to reproduce
\eqref{transfg}
when the background metric $\bg$ is fixed:
\be
\delta^{(Q)}_\epsilon \bg_{\mu\nu}=0\ ;
\qquad
\bg_{\mu\rho}\delta^{(Q)}_\epsilon (e^h)^\rho{}_\nu
=\Lie_\epsilon g_{\mu\nu}\ .
\ee
From the properties of the Lie derivative we obtain:
\be
\Lie_\epsilon g_{\mu\nu}=\Lie_\epsilon \bg_{\mu\rho} (e^h)^\rho{}_\nu
+\bg_{\mu\rho}\Lie_\epsilon(e^h)^\rho{}_\nu
=(\bnabla_\rho\epsilon_\mu+\bnabla_\mu\epsilon_\rho) (e^h)^\rho{}_\nu
+g_{\mu\lambda}(e^{-h})^\lambda{}_\rho\Lie_\epsilon(e^h)^\rho{}_\nu
\ee
and we find:
\be
(e^{-h}\delta^{(Q)}_\epsilon e^h)^\mu{}_\nu=
(e^{-h}\Lie_\epsilon e^h)^\mu{}_\nu
+(e^{-h})^\mu{}_\rho(\bnabla^\rho\epsilon_\sigma+\bnabla_\sigma\epsilon^\rho)
(e^h)^\sigma{}_\nu
\ee
Expanding the latter expression for small values of the quantum fluctuation $h$ the result we obtain is:
\be
\delta^{(Q)}_\epsilon h^\mu{}_\nu=
\bnabla^\mu\epsilon_\nu+\bnabla_\nu\epsilon^\mu
+\Lie_\epsilon h^\mu{}_\nu
+[\Lie_\epsilon\bar g,h]^\mu{}_\nu
+O(\epsilon h^2)\ .
\ee



















