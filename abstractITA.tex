\chapter*{\centering \begin{normalsize}Abstract (italiano)\end{normalsize}}
\begin{quotation}
\noindent
In questa tesi sono state applicate le tecniche del gruppo di rinormalizzazione funzionale allo studio della teoria quantistica di campo scalare con simmetria $O(N)$ sia in uno spaziotempo piatto (Euclideo) che nel caso di accoppiamento ad un campo gravitazionale nel paradigma dell'\emph{asymptotic safety}. 

Nel primo capitolo vengono esposti in breve alcuni concetti basilari della teoria dei campi in uno spazio euclideo a dimensione arbitraria.

Nel secondo capitolo si discute estensivamente il metodo di rinormalizzazione funzionale ideato da Wetterich e si fornisce un primo semplice esempio di applicazione, il modello scalare. 

Nel terzo capitolo \`e stato studiato in dettaglio il modello $O(N)$ in uno spaziotempo piatto, ricavando analiticamente le equazioni di evoluzione delle quantit\`a rilevanti del modello. Quindi ci si \`e specializzati sul caso $N\to\infty$.

Nel quarto capitolo viene iniziata l'analisi delle equazioni di punto fisso nel limite $N \to \infty$, a partire dal caso di dimensione anomala nulla e rinormalizzazione della funzione d'onda costante (approssimazione LPA), gi\`a studiato in letteratura. Viene poi considerato il caso NLO nella derivative expansion.

Nel quinto capitolo si \`e introdotto l'accoppiamento non minimale con un campo gravitazionale, la cui natura quantistica \`e considerata a livello di QFT secondo il paradigma di rinormalizzabilit\`a dell'\emph{asymptotic safety}. Per questo modello si sono ricavate le equazioni di punto fisso per le principali osservabili e se ne \`e studiato il comportamento per diversi valori di $N$.



\end{quotation}
\clearpage